\chapter{总结与展望}
本文研究了隐式几何表达与深度学习框架的结合形式,首先通过编码器结构将输入转化到特征空间,再以特定的隐式几何表达作为中间件,根据不同需求计算输出。本文通过自由视角插值和点云重建这两个具体任务,讨论了该框架融合先验知识和数据驱动的优势及潜力。

环形三相机插值方法能够对大致分布在圆环上的三个相机的图片进行连续插值,革新了原先的两相机之间的视角插值模型。通过环形校正将三张图片对齐到标准圆弧上,并将图片插值过程全部以可导的神经网络表达。选择视差作为几何表达,并辅以可见区域信息。网络的编码器模块提取彩色图片中的特征后,解码器预测视差和可见区域信息,最终渲染模块根据视差的几何意义生成新视角的图片。在一般物体模型和人体模型数据上的实验表明,本方法的新视角生成效果优于其他方法,能够较好地消除遮挡造成的歧义。利用新的视角插值方法能够轻易地生成“子弹时间”特效。

多视角投影的点云重建方法能从粗糙的点云数据中恢复几何,针对点云数据的噪声以及空洞缺陷,能够较好地去噪和补全。点云经过多视角投影得到深度图,深度图经过编码器得到特征张量。在这个任务中以隐函数作为几何的表达形式,能对空间任意点预测隐函数值,最后提取隐函数的等值面。误差函数将几何误差的衡量表达为对采样点的分类误差,将隐函数的拟合与点采样结合起来。在两个人体数据集上的实验表明,新方法处理噪声和空洞的能力明显优于传统方法,多视角投影的编码方式不仅能提取特征,还能利用不同尺度的信息对点云在2D上进行去噪和补全。在真实的Kinect数据上的实验证明了新方法良好的泛化性能。

通过视角插值和点云重建这两个任务,验证了将隐式几何形式内嵌到深度学习框架这一想法的可行性,通过对数据集的选择,也能实现将数据先验和知识先验结合的目标,但是从实验中也发现了一些明显的不足,这主要集中在生成的数据质量、可编辑性以及可解释性方面。

在视角插值任务中,在单张11G显卡的限制下,生成图片尺寸最高只能达到512分辨率,这不满足当前手机等显示设备的高清要求,同时生成图片仍然存在细节模糊不清的缺点。在连续生成图片序列时,有时会出现某一帧质量明显下降,甚至出现明显瑕疵的情况。与传统确定性算法不同,网络生成的数据出现瑕疵时,很难精确定位到具体网络参数,是神经网络可编辑性差的一个体现。

在点云重建任务中,生成的模型往往趋于平滑,对于精细的细节如手指、头发丝以及面部五官细节等,无法获得精确几何。提取隐函数的等值面时,规范网格的分辨率一般为256,最高可达512,但是提取单个模型的时间将需要几分钟,同时提取算法无法根据真实几何是否具备高频细节,自适应地调节需要的采样粒度。对于人体模型,当头发较长,并且距离面部较近时,点云重建结果中头发与面部往往会糊成一团,失去大量细节,对不同的模型表现会有所差异。对这样的问题,同样很难精确定位到网络参数并对其进行编辑。

这些在不同应用中表现出的问题之间存在着明显的相关性。生成的数据质量上呈现出的模糊和细节缺失的问题,应与编码器以卷积操作为主有关。而无规律出现瑕疵的情形,在几乎所有的深度学习任务中都存在。针对细节缺失的问题,需要考虑改良卷积操作,目前有一些工作用基于光线采样的表达形式替换了全图卷积操作,有望能提升细节。针对可编辑性的操作,同样需要改进特征的表达方式,很多产生式模型的工作都涉及讨论隐编码(latent code)对特征空间的结构和可编辑性的作用,借鉴这方面的工作有望能厘清特征空间的更多性质,从而在小的变化范围内定位瑕疵的原因并试图改进。

